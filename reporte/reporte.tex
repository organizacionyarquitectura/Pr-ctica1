%Margenes, idioma y tipo de documento
\documentclass{article}
\usepackage[spanish]{babel}
\usepackage[utf8]{inputenc}
\usepackage[margin = 1.5cm]{geometry}

\begin{document}
    
    %Titulo
    \title{Organización y Arquitectura de Computadoras\\
    \large 2019-2 \\
    \large Práctica 1: Medidas de desempeño}

    \date{Fecha de entrega: 17 de febrero del 2019}

    \author{Sandra del Mar Soto Corderi\\
    Edgar Quiroz Castañeda}

    \maketitle

    %Problemas


    \section{Ejerccios}

    \begin{enumerate}
        %1
        \item {
            Identifica cuales de las pruebas miden el tiempo de respuesta y 
            cuales miden el rendimiento.
        }
        \item {
            Usando la medida de tendencia central adecuada y tu reporte de 
            resultados, calcula
            \begin{itemize}
                \item {
                    La medida de tiemp de respuesta.
                }
                \item {
                    La medidad de rendimiento.
                }
            \end{itemize}
            
        }
        \item {
            Calcula los tiempos normalizados y obtén la medida de tendencia 
            central adecuada de cada una de las computadoras.
        }
        \item {
            Plantea un caso de uso para una computadora. De acuerdo a los 
            requerimientos del usuario, pondera loas resultados de las pruebas 
            y obtén la medida de desempeño de cada una de las computadoras de tu
            equipo.
        }
    \end{enumerate}

    \section{Preguntas}

    \begin{enumerate}
        \item {
            ¿Cuál computadora tiene el mejor tiempo de ejecución? ¿Porqué 
        factor es mejor computadora comparado a la peor?
        }
        \item {
            ¿Cuál computadora tiene el mejor desempeño? ¿Porqué 
        factor es mejor la computadora comparado a la peor?
        }
        \item {
            De acuerdo a la referencia ¿cuál computadora tiene el mejor desempeño
            y cuál tiene el peor desempeño?
        }
        \item {
            ¿Cuál computadora tiene el mejor desempeño para el usuario planteado
            en el caso de uso?
        }
        \item {
            De los atributos de cada máquina, ¿cuáles resultan determinantes en 
            la pérdida o ganancia de desempeño?
        }
    \end{enumerate}

\end{document}