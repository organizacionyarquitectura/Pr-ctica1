% Margenes, idioma y tipo de documento
\documentclass{article}
\usepackage[spanish]{babel}
\usepackage[utf8]{inputenc}
\usepackage[margin = 1.5cm]{geometry}

% Descripciones para las tablas
\usepackage{caption}

\begin{document}
    
    %Titulo
    \title{Organización y Arquitectura de Computadoras\\
    \large 2019-2 \\
    \large Práctica 1: Medidas de desempeño}

    \date{Fecha de entrega: 17 de febrero del 2019}

    \author{Sandra del Mar Soto Corderi\\
    Edgar Quiroz Castañeda}

    \maketitle

    %Problemas


    \section{Ejerccios}

    \begin{enumerate}
        %1
        \item {
            Identifica cuales de las pruebas miden el tiempo de respuesta y 
            cuales miden el rendimiento.

            \begin{itemize}
                \item {
                    GZip Compression \\
                    Description: This test measures the time needed to 
                    archive/compress two copies of the Linux 4.13 kernel source 
                    tree using Gzip compression.\\
                    Por lo que es una prueba de tiempo de respuesta.

                }
                \item {
                    DCRAW \\
                    Description: This test times how long it takes to convert 
                    several high-resolution RAW NEF image files to PPM image 
                    format using dcraw.\\
                    Por lo que es una prueba de tiempo de respuesta.
                }
                \item {
                    FLAC Audio Encoding \\
                    Description: This test times how long it takes to encode a 
                    sample WAV file to FLAC format five times.\\
                    Por lo que es una prueba de tiempo de respuesta.
                }
                \item {
                    GnuPG \\
                    Description: This test times how long it takes to encrypt 
                    a file using GnuPG. \\
                    Por lo que es una prueba de tiempo de respuesta.
                }
                \item {
                    REDIS \\
                    Description: Redis is an open-source data structure server.\\
                    Como es un servidor, probablemente la prueba sea de 
                    rendimiento, donde mide la cantidad de peticiones en un 
                    determinado tiempo.
                }
                \item {
                    MAFFT \\
                    Description: This test performs an alignment of 100 
                    pyruvate decarboxylase sequences.\\
                    Por lo que es una prueba de tiempo de respuesta.
                }
                \item {
                    Bayes Analysis \\
                    Description: This test performs a bayesian analysis of a 
                    set of primate genome sequences in order to estimate their 
                    phylogeny.\\
                    Por lo que es una prueba de tiempo de respuesta.
                }
                \item {
                    MPlayer \\
                    Description: This test times how long it takes to build 
                    the MPlayer media player program.\\
                    Por lo que es una prueba de tiempo de respuesta.
                }
                \item {
                    PHP \\
                    Description: This test times how long it takes to build PHP
                    5 with the Zend engine. \\
                    Por lo que es una prueba de tiempo de respuesta.

                }
            \end{itemize}
        }
        \item {
            Usando la medida de tendencia central adecuada y tu reporte de 
            resultados, calcula
            \begin{itemize}
                \item {
                    La medida de tiempo de respuesta.

                    \begin{table}[]
                        \caption*{Datos de tiempo de respuesta}
                        \begin{center}
                            \begin{tabular}{|l|l|l|}
                                \hline
                                ID           & 1       & 2       \\ \hline
                                REDIS\_LPOP  & 1211506 & 2115110 \\ \hline
                                REDIS\_SADD  & 1027961 & 1710230 \\ \hline
                                REDIS\_LPUSH & 870849  & 1342941 \\ \hline
                                REDIS\_GET   & 1361435 & 2133463 \\ \hline
                                REDIS\_SET   & 951458  & 1492232 \\ \hline
                                Media arit   & 1084641.8 & 1758795.2 \\ \hline
                            \end{tabular}
                        \end{center}
                    \end{table}
                }
                \item {
                    La medida de rendimiento.\\

                    \begin{table}[]
                        \caption*{Datos de rendimiento}
                        \begin{center}
                            \begin{tabular}{|l|l|l|}
                                \hline
                                ID           & 1       & 2       \\ \hline
                                GZip         & 59.49   & 46.43   \\ \hline
                                DCRAW        & 67.86   & 53.56   \\ \hline
                                FLAC         & 19.93   & 13.49   \\ \hline
                                GnuPG        & 18.99   & 14.26   \\ \hline
                                MAFFT        & 11.21   & 8.73    \\ \hline
                                MrBayes      & 763     & 625     \\ \hline
                                MPlayer      & 3.49    & 2.84    \\ \hline
                                PHP          & 295     & 218     \\ \hline
                                Media arm & 15.53 & 12.17 \\ \hline
                            \end{tabular}
                        \end{center}
                    \end{table} 
                }
            \end{itemize}
            
        }
        \item {
            Calcula los tiempos normalizados y obtén la medida de tendencia 
            central adecuada de cada una de las computadoras.
        }
        \item {
            Plantea un caso de uso para una computadora. De acuerdo a los 
            requerimientos del usuario, pondera loas resultados de las pruebas 
            y obtén la medida de desempeño de cada una de las computadoras de tu
            equipo.
        }
    \end{enumerate}

    \section{Preguntas}

    \begin{enumerate}
        \item {
            ¿Cuál computadora tiene el mejor tiempo de ejecución? ¿Porqué 
        factor es mejor computadora comparado a la peor?
        }
        \item {
            ¿Cuál computadora tiene el mejor desempeño? ¿Porqué 
        factor es mejor la computadora comparado a la peor?
        }
        \item {
            De acuerdo a la referencia ¿cuál computadora tiene el mejor desempeño
            y cuál tiene el peor desempeño?
        }
        \item {
            ¿Cuál computadora tiene el mejor desempeño para el usuario planteado
            en el caso de uso?
        }
        \item {
            De los atributos de cada máquina, ¿cuáles resultan determinantes en 
            la pérdida o ganancia de desempeño?
        }
    \end{enumerate}

\end{document}