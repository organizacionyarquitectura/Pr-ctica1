% Margenes, idioma y tipo de documento
\documentclass{article}
\usepackage[spanish]{babel}
\usepackage[utf8]{inputenc}
\usepackage[margin = 1.5cm]{geometry}
\usepackage{booktabs}
\usepackage[table,xcdraw]{xcolor}
\usepackage{tabularx}
\newcolumntype{L}{>{\raggedright\arraybackslash}X}

% Descripciones para las tablas
\usepackage{caption}

% Controlar posición de las tablas
\usepackage{float}

\begin{document}
    
    %Titulo
    \title{Organización y Arquitectura de Computadoras\\
    \large 2019-2 \\
    \large Práctica 1: Medidas de desempeño}

    \date{Fecha de entrega: 17 de febrero del 2019}

    \author{Sandra del Mar Soto Corderi\\
    Edgar Quiroz Castañeda}

    \maketitle

    %Resultados
    
    %Tabla de propiedades de computadoras
\begin{table}[H]
\caption*{Propiedades de las computadoras utilizadas}
\begin{tabularx}{\linewidth}{|L|L|L|L|L|}
\hline
Propiedad/Computadora                                   & \cellcolor[HTML]{F8A102}{\color[HTML]{000000} A(Alan)} & \cellcolor[HTML]{FFFE65}{\color[HTML]{000000} B(César)}     & \cellcolor[HTML]{34FF34}{\color[HTML]{000000} C(Edgar)}                & \cellcolor[HTML]{9698ED}{\color[HTML]{000000} D(Sandra)} \\ \hline
\cellcolor[HTML]{DAE8FC}Nombre del alumno        & Alan Ernesto Arteaga	Vázquez                              &  	César Hernández Cruz                            & Edgar Quiróz	Castañeda  & Sandra del Mar Soto Corderí  \\ \hline
\cellcolor[HTML]{DAE8FC}Motherboard (Fabricante, modelo y BIOS)           & HP 220F v57.51(F.36 BIOS)                              & LENOVO Guam (36CN17WWV2.03 BIOS)                            & Acer ZORO BH (V1.37 BIOS)                                              & Dell 06K7YG (1.7.5 BIOS)                                 \\ \hline
\cellcolor[HTML]{DAE8FC}Procesador (Fabricante, modelo, frecuencia, núcleos)             & Intel Celeron N2840 @ 2.58GHz (2 Núcleos)                & AMD Athlon II P340 @ 2.20GhZ (2 Núcleos)                      & Intel Core i5-5200U @ 2.70GHz (2 Núcleos/ 4 Threads)                     & Intel Core i5-7200U @ 3.10GHz (2 Núcleos/ 4 Threads)       \\ \hline
\cellcolor[HTML]{DAE8FC}Memoria (RAM y Caché de los procesadores)                & 4096MB/ 1024kb                                                & 8192MB/ 512kb                                                     & 12288MB/ 3072kb                                                                & 16384MB/ 3072kb                                                  \\ \hline
\cellcolor[HTML]{DAE8FC}Disco Duro (Capacidad, tipo y velocidad)                   & 500GB/ Seagate ST500LT012-1DG14/ 5400 rpm                         & 250GB/ Samsung SSD 860/ 550MBs Lectura y 520MBs Escritura                                       & 1000GB/ TOSHIBA MQ01ABD1/ 5400 rpm                                                & 500GB/ Seagate ST500LM021-1KJ15/ 7200 rpm                           \\ \hline
\cellcolor[HTML]{DAE8FC}Distribución de linux                     & elementary 5.0                                         & Ubuntu 18.04                                                & Fedora 29                                                              & Ubuntu 18.04                                             \\ \hline
\cellcolor[HTML]{DAE8FC}Kernel                 & 4.15.0-36-generic (x86\_64)                            & 4.15.0-45-generic (x86\_64)                                 & 4.20.4-200.fc29.x86\_64 (x86\_64)                                      & 4.15.0-45-generic (x86\_64)                              \\ \hline
\cellcolor[HTML]{DAE8FC}Chipset                & Intel Atom Z36xxx/Z37xxx                               & AMD RS880                                                   & Intel Broadwell-U-OPI                                                  & Intel Xeon E3-1200 v6/7th                                \\ \hline
\cellcolor[HTML]{DAE8FC}Graficos               & Intel Atom Z36xxx/Z37xxx \& Display (792MHz)           & AMD Mobility Radeon HD 4225/4250 256MB                      & Intel HD 5500 3072MB (900MHz)                                          & Intel HD 620 (1000MHz)                                   \\ \hline
\cellcolor[HTML]{DAE8FC}Audio                  & Realtek ALC3227                                        & Realtek ALC259                                              & Intel Broadwell-U Audio                                                & Realtek ALC3246                                          \\ \hline
\cellcolor[HTML]{DAE8FC}Network                & Realtek RTL8101/2/6E + Qualcomm Atheros AR9485         & Qualcomm Atheros AR8152 v1.1 Fast + Qualcomm Atheros AR9285 & Realtek RTL8111/8168/8411 + Qualcomm Atheros QCA9377 802.11ac Wireless & Intel I219-LM + Qualcomm Atheros QCA6174 802.11ac        \\ \hline
\cellcolor[HTML]{DAE8FC}Display Server         & X Server 1.19.6                                        & X Server 1.19.6                                             & X Server 1.20.3                                                        & X Server 1.19.6                                          \\ \hline
\cellcolor[HTML]{DAE8FC}Display Driver         & modesetting 1.19.6                                     & modesetting 1.19.6                                          & modesetting 1.20.3                                                     & modesetting 1.19.6                                       \\ \hline
\cellcolor[HTML]{DAE8FC}Compilador             & GCC 7.3.0 + Clang 6.0.0-1ubuntu2                       & GCC 7.3.0                                                   & GCC 8.2.1 20181515                                                     & GCC 7.3.0                                                \\ \hline
\cellcolor[HTML]{DAE8FC}Sistema de Archivos    & ext4                                                   & ext4                                                        & ext4                                                                   & ext4                                                     \\ \hline
\cellcolor[HTML]{DAE8FC}Resolución de pantalla & 1366x768                                               & 1366x768                                                    & 1366x768                                                               & 1366x768                                                 \\ \hline
\end{tabularx}
\end{table}


    \section{Ejercicios}

    \begin{enumerate}
        %1
        \item {
            Identifica cuales de las pruebas miden el tiempo de respuesta y 
            cuales miden el rendimiento.

            \begin{itemize}
                \item {
                    GZip Compression \\
                    Descripción: Esta prueba mide el tiempo necesario para archivar/comprimir dos copias del árbol de fuentes del núcleo Linux 4.13 usando compresión Gzip. \\
                    Por lo que es una prueba de $\bf{tiempo\ de\ respuesta}$.

                }
                \item {
                    DCRAW \\
                    Descripción: Esta prueba calcula cuánto tiempo tarda para convertir varios archivos de imagen RAW NEF alta resolución en formato de imagen PPM usando dcraw. \\
                    Por lo que es una prueba de $\bf{tiempo\ de\ respuesta}$.
                }
                \item {
                    FLAC Audio Encoding \\
                    Esta prueba calcula el tiempo que tarda para codificar un archivo WAV a formato FLAC cinco veces.\\
                    Por lo que es una prueba de $\bf{tiempo\ de\ respuesta}$.
                }
                \item {
                    GnuPG \\
                    Descripción: Esta prueba calcula el tiempo que tarda para cifrar un archivo con GnuPG.  \\
                    Por lo que es una prueba de $\bf{tiempo\ de\ respuesta}$.
                }
                \item {
                    REDIS \\
                   Descripción: Redis es un servidor de estructura de datos de código abierto.\\
                    Como es un servidor, la prueba es de $\bf{rendimiento}$, donde mide la cantidad de peticiones en un 
                    determinado tiempo.
                }
                \item {
                    MAFFT \\
                    Descripción: Esta prueba calcula cuánto tiempo tarda realizar una alineación de 100 secuencias de decarboxilasa piruvato.\\
                    Por lo que es una prueba de $\bf{tiempo\ de\ respuesta}$.
                }
                \item {
                    Bayes Analysis \\
                    Descripción: Esta prueba calcula el tiempo que tarda realizar un análisis bayesiano de un conjunto de secuencias de genoma de primates con el fin de estimar su filogenia.\\
                    Por lo que es una prueba de $\bf{tiempo\ de\ respuesta}$.
                }
                \item {
                    MPlayer \\
                    Descripción: Esta prueba calcula el tiempo que tarda para construir el programa de reproductor multimedia MPlayer. \\
                    Por lo que es una prueba de $\bf{tiempo\ de\ respuesta}$.
                }
                \item {
                    PHP \\
                    Descripción: Esta prueba calcula el tiempo que tarda para construir PHP 5 con el motor Zend. \\
                    Por lo que es una prueba de $\bf{tiempo\ de\ respuesta}$.\\

                }
            \end{itemize}
        }
        \item {
            Usando la medida de tendencia central adecuada y tu reporte de 
            resultados, calcula            
            \begin{itemize}
                \item {
                    La medida de tiempo de respuesta.
                    \begin{table}[H]
                        \caption*{Datos de tiempo de respuesta}
                        \begin{tabular}{|l|l|l|l|l|l|l|l|l|l|}
                        \toprule
                        Computadora / Pruebas
                        & \cellcolor[HTML]{DAE8FC}build-mplayer 
                        & \cellcolor[HTML]{DAE8FC}build-php 
                        & \cellcolor[HTML]{DAE8FC}gzip 
                        & \cellcolor[HTML]{DAE8FC}dcraw  
                        & \cellcolor[HTML]{DAE8FC}flac 
                        & \cellcolor[HTML]{DAE8FC}gnupg 
                        & \cellcolor[HTML]{DAE8FC}mafft 
                        & \cellcolor[HTML]{DAE8FC}mrbayes 
                        & \cellcolor[HTML]{DAE8FC}media armónica \\ \hline
            
                        \cellcolor[HTML]{F8A102}{\color[HTML]{000000}} 
                        A (Alan) & 422.32 & 831.64 & 96.77 & 201.06 & 44.19 
                        & 37.29 & 27.64 & 2548.70 & 76.27\\ \hline
            
                        \cellcolor[HTML]{FFFE65}{\color[HTML]{000000}}
                        B (César) & 5.84 & 547.82 & 78.23 & 160.81 & 50.04 
                        & 41.71 & 24.60 & 1886.59 & 28.86\\ \hline
            
                        \cellcolor[HTML]{34FF34}{\color[HTML]{000000}}
                        C (Edgar) & 3.49 & 295 & 59.49 & 67.86 & 19.93 & 18.99 
                        & 11.21 & 762.99 & 15.54\\ \hline
            
                        \cellcolor[HTML]{9698ED}{\color[HTML]{000000} 
                        D(Sandra)} & 2.84 & 217.87 & 47.53 & 53.56 & 13.49 
                        & 14.26 & 8.73 & 625.15 & 12.19   \\ \hline
                        \end{tabular}
                    \end{table}
                }
                
                \item {
                    La medida de rendimiento.

                    \begin{table}[H]
                        \caption*{Datos de rendimiento}
                        \begin{tabular}{|l|l|l|l|l|l|l|}
                        \toprule
                            Computadora / Pruebas 
                            & \cellcolor[HTML]{DAE8FC}redis(LPOP) 
                            & \cellcolor[HTML]{DAE8FC}redis(SADD) 
                            & \cellcolor[HTML]{DAE8FC}redis(LPUSH) 
                            & \cellcolor[HTML]{DAE8FC}redis(GET)
                            & \cellcolor[HTML]{DAE8FC}redis(SET) 
                            & \cellcolor[HTML]{DAE8FC}media aritmética\\ \hline
            
                            \cellcolor[HTML]{F8A102}{\color[HTML]{000000}} 
                            A (Alan) & 553354 & 429677 & 307946 & 500548 
                            & 362198 & 430744.6\\ \hline
            
                            \cellcolor[HTML]{FFFE65}{\color[HTML]{000000}}
                            B (César) & 988937.98 & 734887.83 & 489798.29 
                            & 936797.44 & 644343.29 & 758952.97\\ \hline
            
                            \cellcolor[HTML]{34FF34}{\color[HTML]{000000}} 
                            C (Edgar) & 1211505.83 & 1027960.93 & 870849.42 
                            & 1361435 & 951458 & 1084641.84\\ \hline
            
                            \cellcolor[HTML]{9698ED}{\color[HTML]{000000} 
                            D(Sandra)} & 2115109.73 & 1710230.42 & 1342941.46 
                            & 2133463 & 1492232 & 1758795.32 \\ \hline
                        \end{tabular}
                    \end{table} 
                }
            \end{itemize}
            
        }
        \item {
            Calcula los tiempos normalizados y obtén la medida de tendencia 
            central adecuada de cada una de las computadoras. \\
            
            Los datos estarán normalizados respecto a la computadora A, la cual fue escogida arbitrariamente.

            \begin{table}[H]
                \caption*{Datos de tiempo de respuesta normalizados respecto a A}
                \begin{tabular}{|l|l|l|l|l|l|l|l|l|l|}
                \toprule
                Computadora / Pruebas 
                & \cellcolor[HTML]{DAE8FC}build-mplayer 
                & \cellcolor[HTML]{DAE8FC}build-php 
                & \cellcolor[HTML]{DAE8FC}gzip 
                & \cellcolor[HTML]{DAE8FC}dcraw  
                & \cellcolor[HTML]{DAE8FC}flac 
                & \cellcolor[HTML]{DAE8FC}gnupg 
                & \cellcolor[HTML]{DAE8FC}mafft 
                & \cellcolor[HTML]{DAE8FC}mrbayes 
                & \cellcolor[HTML]{DAE8FC}media geométrica \\ \hline
    
                \cellcolor[HTML]{F8A102}{\color[HTML]{000000}} 
                A (Alan) & 1 & 1 & 1 & 1 & 1 & 1 & 1 & 1 & 1 \\ \hline
    
                \cellcolor[HTML]{FFFE65}{\color[HTML]{000000}}
                B (César) & 0.01382 & 0.65872 & 0.80841 & 0.79981 & 1.1323 
                & 1.1185 & 0.89001 & 0.7402 & 0.51456 \\ \hline
    
                
    
                \cellcolor[HTML]{34FF34}{\color[HTML]{000000}}
                C (Edgar) & 0.0082638 & 0.35472 & 0.61475 & 0.33751 & 0.45100 
                & 0.50925 & 0.40557 & 0.29936 & 0.25332\\ \hline
    
                
    
                \cellcolor[HTML]{9698ED}{\color[HTML]{000000} 
                D(Sandra)} & 0.0067247 & 0.26197 & 0.49116 & 0.26638 & 0.30527 
                & 0.38240 & 0.31584 & 0.24528 & 0.19493 \\ \hline
    
                
                \end{tabular}
            \end{table}
    
            \begin{table}[H]
                \caption*{Datos de rendimiento normalizados respecto a A}
                \begin{tabular}{|l|l|l|l|l|l|l|}
                \toprule
                    Computadora / Pruebas
                    & \cellcolor[HTML]{DAE8FC}redis(LPOP) 
                    & \cellcolor[HTML]{DAE8FC}redis(SADD) 
                    & \cellcolor[HTML]{DAE8FC}redis(LPUSH) 
                    & \cellcolor[HTML]{DAE8FC}redis(GET)
                    & \cellcolor[HTML]{DAE8FC}redis(SET) 
                    & \cellcolor[HTML]{DAE8FC}media geométrica \\ \hline
    
                    \cellcolor[HTML]{F8A102}{\color[HTML]{000000}} 
                    A (Alan) & 1 & 1 & 1 & 1 & 1 & 1 \\ \hline
    
                    \cellcolor[HTML]{FFFE65}{\color[HTML]{000000}}
                    B (César) & 1.787170564 & 1.710326199 & 1.590533048 
                    & 1.871543668 & 1.778980806 & 1.745146036\\ \hline
    
    
                    \cellcolor[HTML]{34FF34}{\color[HTML]{000000}} 
                    C (Edgar) & 2.189386595 & 2.392403899 & 2.827928988 
                    & 2.719889002 & 2.626900204 & 2.54052833\\ \hline
    
                    \cellcolor[HTML]{9698ED}{\color[HTML]{000000} 
                    D(Sandra)} & 3.822344702 & 3.980269877 & 4.36096413 
                    & 4.262254569 & 4.119934401 & 4.104600691 \\ \hline
    
                 
                \end{tabular}
            \end{table} 
        }
        \item {
            Plantea un caso de uso para una computadora. De acuerdo a los 
            requerimientos del usuario, pondera los resultados de las pruebas 
            y obtén la medida de desempeño de cada una de las computadoras de tu
            equipo. \\
            
            Como caso de uso se propone un desarrollador de software que necesita
            una computadora personal. \\
            En este caso no es tan relevante que tenga buen desempeño en 
            analisis de datos, por lo que el analisis bayesiano y el analisis 
            genético deberían tener poco peso. \\
            Herramientas para manipulación de media y archivos son medianamente 
            importante, pues aunque son indispensables es suficiente con que 
            funciones aceptablemente. \\
            La parte más importante serían las pruebas de compilación, pues es 
            básicamente para lo que se va a usar la computadora. \\
            En la parte de desempeño, no se requiere que se pueda manejar grandes 
            cantidades de peticiones, pues el equipo no está pensado para usarse
            como servidor. Aún así, sería conveniente que tenga un buen desempeño
            al realizar peticiones, pues esto puede ser útil al realizar pruebas
            de conexión entre la interfaz de usuario y algún servidor.

            \begin{table}[H]
                \caption*{Pesos para los tiempos de respuesta}
                \begin{center}
                    \begin{tabular}{|l|l|l|l|l|l|l|l|l|l|}
                        \toprule
                        Pc / Tareas 
                        & \cellcolor[HTML]{DAE8FC}build-mplayer 
                        & \cellcolor[HTML]{DAE8FC}build-php 
                        & \cellcolor[HTML]{DAE8FC}gzip 
                        & \cellcolor[HTML]{DAE8FC}dcraw  
                        & \cellcolor[HTML]{DAE8FC}flac 
                        & \cellcolor[HTML]{DAE8FC}gnupg 
                        & \cellcolor[HTML]{DAE8FC}mafft 
                        & \cellcolor[HTML]{DAE8FC}mrbayes \\ \hline
            
                        \cellcolor[HTML]{F8A102}{\color[HTML]{000000}} 
                        Pesos & 0.25 & 0.25 & 0.1 & 0.1 & 0.1 & 0.1 & 0.05 
                        & 0.05 \\ \hline
        
                        \end{tabular}
                \end{center}
            \end{table}

			Para obtener la media armónica ponderada seguimos la siguiente fórmula: $\sum_{i=1}^{n} w_{i} \div \sum_{i=1}^{n} (w_{i} \div x_{i})$ \\
			donde w representan los pesos, x los datos y la suma de los pesos es igual a 1.
            \begin{table}[H]
                \caption*{Medidas de tendencia de tiempo de respuesta ponderadas}
                \begin{center}
                    \begin{tabular}{|l|l|l|l|l|l|l|l|l|l|}
                        \toprule
                        Computadora / Pruebas
                        & \cellcolor[HTML]{DAE8FC}Media Armónica Ponderada\\ \hline
            
                        \cellcolor[HTML]{F8A102}{\color[HTML]{000000}} 
                        A (Alan) & 108.736416907889 \\ \hline
            
                        \cellcolor[HTML]{FFFE65}{\color[HTML]{000000}}
                        B (César) & 19.3724720236483 \\ \hline
            
                        \cellcolor[HTML]{34FF34}{\color[HTML]{000000}}
                        C (Edgar) & 11.0564922536646 \\ \hline
            
                        \cellcolor[HTML]{9698ED}{\color[HTML]{000000} 
                        D(Sandra)} & 8.81993499466002 \\ \hline
            
                        \end{tabular}
                \end{center}
            \end{table}

            \begin{table}[H]
                \caption*{Pesos para las pruebas de desempeño}
                \begin{center}
                    \begin{tabular}{|l|l|l|l|l|l|l|}
                        \toprule
                            Pc / Tareas 
                            & \cellcolor[HTML]{DAE8FC}redis(LPOP) 
                            & \cellcolor[HTML]{DAE8FC}redis(SADD) 
                            & \cellcolor[HTML]{DAE8FC}redis(LPUSH) 
                            & \cellcolor[HTML]{DAE8FC}redis(GET)
                            & \cellcolor[HTML]{DAE8FC}redis(SET) \\ \hline
            
                            \cellcolor[HTML]{F8A102}{\color[HTML]{000000}} 
                            Pesos & 0.25 & 0.15 & 0.15 & 0.15 & 0.3 \\ \hline
        
                        \end{tabular}
                \end{center}
            \end{table} 

			Para obtener la media aritmética ponderada seguimos la fórmula vista en clase $\sum_{i=1}^{n} w_{i}\delta_{i}$ donde w representan los pesos, $\delta$ los datos y la suma de los pesos es igual a 1\\

            \begin{table}[H]
                \caption*{Medidas de tendencias de desempeño ponderadas}
                \begin{center}
                    \begin{tabular}{|l|l|l|l|l|l|l|}
                        \toprule
                            Pc / Tareas 
                            & \cellcolor[HTML]{DAE8FC}Media Aritmética Ponderada \\ \hline
            
                            \cellcolor[HTML]{F8A102}{\color[HTML]{000000}} 
                            A (Alan) & 432723.55 \\ \hline
            
                            \cellcolor[HTML]{FFFE65}{\color[HTML]{000000}}
                            B (César) & 764760.016 \\ \hline
            
            
                            \cellcolor[HTML]{34FF34}{\color[HTML]{000000}} 
                            C (Edgar) & 1077350.66\\ \hline
            
                            \cellcolor[HTML]{9698ED}{\color[HTML]{000000} 
                            D(Sandra)} & 1754442.2645 \\ \hline
                    
                        \end{tabular}
                \end{center}
            \end{table} 
        }
    \end{enumerate}

    \section{Preguntas}

    \begin{enumerate}
        \item {
        ¿Cual computadora tiene el mejor tiempo de ejecución?\\
         La computadora D tiene el mejor tiempo de ejecución de acuerdo a nuestras medidas de desempeño.\\
         
         Esto es ya que es la que tiene menor media armónica, esto es porque tomamos el parámetro LIB (Lower Is Better), en el cual mientras más pequeña sea la media mejor es el tiempo de ejecución de la computadora. Usamos la media armónica como base de comparación porque se sesga ante valores muy pequeños y aquí nos importa que sean lo más pequeños posibles.\\
      
        Comparada con la computadora con la peor medida de tiempo de ejecucion ¿por qué factor es mejor la computadora?\\
         El tiempo de ejecución de la computadora D es 6.26 veces menor que el de la computadora D.\\
         
          Porque al dividir sus medias armónicas tenemos $\frac{76.2660601835012}{12.1799911458345} = 6.26158584766984$\\
        }
        
        \item {
        ¿Cual computadora tiene el mejor rendimiento?\\
         La computadora D tiene el mejor rendimiento de acuerdo a nuestras medidas de desempeño.\\
          
         Porque es la que tiene la mayor media aritmética esto es porque tomamos el parámetro BIB (Bigger Is Better), en el cual mientras mayor sea la media mejor es el rendimiento de la computadora. Usamos la media aritmética como base de comparación porque se deja llevar por valores muy grandes y aquí nos importa que sean lo más grandes posibles.\\
        
        Comparada con la computadora con el peor desempeño ¿por qué factor es mejor la computadora?\\
        El rendimiento de la computadora D es 4.08 veces mayor que el de la computadora A.\\
        
        Esto es ya que al dividir sus medias aritméticas tenemos $\frac{1758795.322}{430744.6} = 4.08315118053714$\\
               
        }
       
       
        \item {
        De acuerdo a la computadora de referencia, ¿cuál computadora tiene el
mejor desempeño y cuál tiene el peor desempeño? \\
			Usamos la computadora A de forma arbitraria como referencia.De acuerdo a la computadora A, la computadora D tiene el mejor desempeño y la A tiene el peor desempeño.\\
			
			Porque la media geométrica de D al parametrizar el rendimiento es la mayor mientras que la de la computadora A es el menor y la media geométrica de D al parametrizar el tiempo de respuesta es la menor mientras que la de la computadora A era el mayor. Hay que recordar que usamos LIB para el tiempo y BIB para el rendimiento. También tomamos la media geométrica como base de comparación porque no se sesga ante valores normalizados, los cuales estamos utilizando.\\
        }
        
        \item {
            ¿Cuál computadora tiene el mejor desempeño para el usuario planteado
            en el caso de uso? \\
            La computadora D tiene el mejor desempeño para el usuario.\\
            
          Debido a que su media aritmética ponderada es la mayor y su media armónica ponderada es la menor. Hay que recordar que usamos LIB para la media armónica que mide al tiempo de ejecución y BIB para la media aritmética que mide el rendimiento. Utilizamos medias ponderadas ya que de esta forma podemos estar seguros que el usuario tendrá lo que mejor le convenga de acuerdo a sus necesidades, dandole mayor peso a las pruebas que más necesita.\\
        }
        
        \item {
            De los atributos de cada máquina, ¿cuáles resultan determinantes en 
            la pérdida o ganancia de desempeño? \\
            Los atributos que parecen más influyentes son la cantidad de hilos 
            de ejecución, la frecuencia y la velocidad del procesador y la cantidad de memoria RAM ya que podemos ver que las máquinas que mejor les fue tenían mayor nivel en estas propiedades que a las que peor les fue.
            
            En menor medida son la distribución de linux, kernel, compilador, sistema de archivos, pantalla y chipset ya que todas las computadoras tienen estas propiedades parecidas y no afectan mucho.
        }
    \end{enumerate}
    
    \section{Bibliografía}
    
    

\begin{thebibliography}{9}

\bibitem{armonicaponderada}
Cal, E. (2019). Mathematics and Applied Statistics Lesson of the Day – The Weighted Harmonic Mean.\\ 
Obtenida de https://chemicalstatistician.wordpress.com/2014/06/25/mathematics-and-applied-statistics-lesson-of-the-day-the-weighted-harmonic-mean

\end{thebibliography}


\end{document}