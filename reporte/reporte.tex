% Margenes, idioma y tipo de documento
\documentclass{article}
\usepackage[spanish]{babel}
\usepackage[utf8]{inputenc}
\usepackage[margin = 1.5cm]{geometry}
\usepackage{booktabs}
\usepackage[table,xcdraw]{xcolor}
\usepackage{tabularx}
\newcolumntype{L}{>{\raggedright\arraybackslash}X}

% Descripciones para las tablas
\usepackage{caption}

% Controlar posición de las tablas
\usepackage{float}

\begin{document}
    
    %Titulo
    \title{Organización y Arquitectura de Computadoras\\
    \large 2019-2 \\
    \large Práctica 1: Medidas de desempeño}

    \date{Fecha de entrega: 17 de febrero del 2019}

    \author{Sandra del Mar Soto Corderi\\
    Edgar Quiroz Castañeda}

    \maketitle

    %Resultados
    
    %Tabla de propiedades de computadoras
\begin{table}[H]
\caption*{Propiedades de las computadoras utilizadas}
\begin{tabularx}{\linewidth}{|L|L|L|L|L|}
\hline
Propiedad/Pc                                   & \cellcolor[HTML]{F8A102}{\color[HTML]{000000} A(Alan)} & \cellcolor[HTML]{FFFE65}{\color[HTML]{000000} B(César)}     & \cellcolor[HTML]{34FF34}{\color[HTML]{000000} C(Edgar)}                & \cellcolor[HTML]{9698ED}{\color[HTML]{000000} D(Sandra)} \\ \hline
\cellcolor[HTML]{DAE8FC}Procesador             & Intel Celeron N2840 @ 2.58GHz (2 Cores)                & AMD Athlon II P340 @ 2.20GhZ (2 Cores)                      & Intel Core i5-5200U @ 2.70GHz (2 Cores/ 4 Threads)                     & Intel Core i5-7200U @ 3.10GHz (2 Cores/ 4 Threads)       \\ \hline
\cellcolor[HTML]{DAE8FC}Motherboard            & HP 220F v57.51(F.36 BIOS)                              & LENOVO Guam (36CN17WWV2.03 BIOS)                            & Acer ZORO BH (V1.37 BIOS)                                              & Dell 06K7YG (1.7.5 BIOS)                                 \\ \hline
\cellcolor[HTML]{DAE8FC}Chipset                & Intel Atom Z36xxx/Z37xxx                               & AMD RS880                                                   & Intel Broadwell-U-OPI                                                  & Intel Xeon E3-1200 v6/7th                                \\ \hline
\cellcolor[HTML]{DAE8FC}Memoria                & 4096MB                                                 & 8192MB                                                      & 12288MB                                                                & 16384MB                                                  \\ \hline
\cellcolor[HTML]{DAE8FC}Disk                   & 500GB Seagate ST500LT012-1DG14                         & 250GB Samsung SSD 860                                       & 1000GB TOSHIBA MQ01ABD1                                                & 500GB Seagate ST500LM021-1KJ15                           \\ \hline
\cellcolor[HTML]{DAE8FC}Graficos               & Intel Atom Z36xxx/Z37xxx \& Display (792MHz)           & AMD Mobility Radeon HD 4225/4250 256MB                      & Intel HD 5500 3072MB (900MHz)                                          & Intel HD 620 (1000MHz)                                   \\ \hline
\cellcolor[HTML]{DAE8FC}Audio                  & Realtek ALC3227                                        & Realtek ALC259                                              & Intel Broadwell-U Audio                                                & Realtek ALC3246                                          \\ \hline
\cellcolor[HTML]{DAE8FC}Network                & Realtek RTL8101/2/6E + Qualcomm Atheros AR9485         & Qualcomm Atheros AR8152 v1.1 Fast + Qualcomm Atheros AR9285 & Realtek RTL8111/8168/8411 + Qualcomm Atheros QCA9377 802.11ac Wireless & Intel I219-LM + Qualcomm Atheros QCA6174 802.11ac        \\ \hline
\cellcolor[HTML]{DAE8FC}OS                     & elementary 5.0                                         & Ubuntu 18.04                                                & Fedora 29                                                              & Ubuntu 18.04                                             \\ \hline
\cellcolor[HTML]{DAE8FC}Kernel                 & 4.15.0-36-generic (x86\_64)                            & 4.15.0-45-generic (x86\_64)                                 & 4.20.4-200.fc29.x86\_64 (x86\_64)                                      & 4.15.0-45-generic (x86\_64)                              \\ \hline
\cellcolor[HTML]{DAE8FC}Display Server         & X Server 1.19.6                                        & X Server 1.19.6                                             & X Server 1.20.3                                                        & X Server 1.19.6                                          \\ \hline
\cellcolor[HTML]{DAE8FC}Display Driver         & modesetting 1.19.6                                     & modesetting 1.19.6                                          & modesetting 1.20.3                                                     & modesetting 1.19.6                                       \\ \hline
\cellcolor[HTML]{DAE8FC}Compilador             & GCC 7.3.0 + Clang 6.0.0-1ubuntu2                       & GCC 7.3.0                                                   & GCC 8.2.1 20181515                                                     & GCC 7.3.0                                                \\ \hline
\cellcolor[HTML]{DAE8FC}Sistema de Archivos    & ext4                                                   & ext4                                                        & ext4                                                                   & ext4                                                     \\ \hline
\cellcolor[HTML]{DAE8FC}Resolución de pantalla & 1366x768                                               & 1366x768                                                    & 1366x768                                                               & 1366x768                                                 \\ \hline
\end{tabularx}
\end{table}


    \section{Ejercicios}

    \begin{enumerate}
        %1
        \item {
            Identifica cuales de las pruebas miden el tiempo de respuesta y 
            cuales miden el rendimiento.

            \begin{itemize}
                \item {
                    GZip Compression \\
                    Description: This test measures the time needed to 
                    archive/compress two copies of the Linux 4.13 kernel source 
                    tree using Gzip compression.\\
                    Por lo que es una prueba de tiempo de respuesta.

                }
                \item {
                    DCRAW \\
                    Description: This test times how long it takes to convert 
                    several high-resolution RAW NEF image files to PPM image 
                    format using dcraw.\\
                    Por lo que es una prueba de tiempo de respuesta.
                }
                \item {
                    FLAC Audio Encoding \\
                    Description: This test times how long it takes to encode a 
                    sample WAV file to FLAC format five times.\\
                    Por lo que es una prueba de tiempo de respuesta.
                }
                \item {
                    GnuPG \\
                    Description: This test times how long it takes to encrypt 
                    a file using GnuPG. \\
                    Por lo que es una prueba de tiempo de respuesta.
                }
                \item {
                    REDIS \\
                    Description: Redis is an open-source data structure server.\\
                    Como es un servidor, probablemente la prueba sea de 
                    rendimiento, donde mide la cantidad de peticiones en un 
                    determinado tiempo.
                }
                \item {
                    MAFFT \\
                    Description: This test performs an alignment of 100 
                    pyruvate decarboxylase sequences.\\
                    Por lo que es una prueba de tiempo de respuesta.
                }
                \item {
                    Bayes Analysis \\
                    Description: This test performs a bayesian analysis of a 
                    set of primate genome sequences in order to estimate their 
                    phylogeny.\\
                    Por lo que es una prueba de tiempo de respuesta.
                }
                \item {
                    MPlayer \\
                    Description: This test times how long it takes to build 
                    the MPlayer media player program.\\
                    Por lo que es una prueba de tiempo de respuesta.
                }
                \item {
                    PHP \\
                    Description: This test times how long it takes to build PHP
                    5 with the Zend engine. \\
                    Por lo que es una prueba de tiempo de respuesta.

                }
            \end{itemize}
        }
        \item {
            Usando la medida de tendencia central adecuada y tu reporte de 
            resultados, calcula
            \begin{itemize}
                \item {
                    La medida de tiempo de respuesta.
                    \begin{table}[H]
                        \caption*{Datos de tiempo de respuesta}
                        \begin{tabular}{|l|l|l|l|l|l|l|l|l|l|}
                        \toprule
                        Pc / Tareas 
                        & \cellcolor[HTML]{DAE8FC}build-mplayer 
                        & \cellcolor[HTML]{DAE8FC}build-php 
                        & \cellcolor[HTML]{DAE8FC}gzip 
                        & \cellcolor[HTML]{DAE8FC}dcraw  
                        & \cellcolor[HTML]{DAE8FC}flac 
                        & \cellcolor[HTML]{DAE8FC}gnupg 
                        & \cellcolor[HTML]{DAE8FC}mafft 
                        & \cellcolor[HTML]{DAE8FC}mrbayes 
                        & \cellcolor[HTML]{DAE8FC}media arm \\ \hline
            
                        \cellcolor[HTML]{F8A102}{\color[HTML]{000000}} 
                        A (Alan) & 422.32 & 831.64 & 96.77 & 201.06 & 44.19 
                        & 37.29 & 27.64 & 2548.70 & 76.27\\ \hline
            
                        \cellcolor[HTML]{FFFE65}{\color[HTML]{000000}}
                        B (César) & 5.84 & 547.82 & 78.23 & 160.81 & 50.04 
                        & 41.71 & 24.60 & 1886.59 & 28.86\\ \hline
            
                        \cellcolor[HTML]{34FF34}{\color[HTML]{000000}}
                        C (Edgar) & 3.49 & 295 & 59.49 & 67.86 & 19.93 & 18.99 
                        & 11.21 & 762.99 & 15.54\\ \hline
            
                        \cellcolor[HTML]{9698ED}{\color[HTML]{000000} 
                        D(Sandra)} & 2.84 & 217.87 & 47.53 & 53.56 & 13.49 
                        & 14.26 & 8.73 & 625.15 & 12.19   \\ \hline
                        \end{tabular}
                    \end{table}
                }
                
                \item {
                    La medida de rendimiento.

                    \begin{table}[H]
                        \caption*{Datos de rendimiento}
                        \begin{tabular}{|l|l|l|l|l|l|l|}
                        \toprule
                            Pc / Tareas 
                            & \cellcolor[HTML]{DAE8FC}redis(LPOP) 
                            & \cellcolor[HTML]{DAE8FC}redis(SADD) 
                            & \cellcolor[HTML]{DAE8FC}redis(LPUSH) 
                            & \cellcolor[HTML]{DAE8FC}redis(GET)
                            & \cellcolor[HTML]{DAE8FC}redis(SET) 
                            & \cellcolor[HTML]{DAE8FC}media arm\\ \hline
            
                            \cellcolor[HTML]{F8A102}{\color[HTML]{000000}} 
                            A (Alan) & 553354 & 429677 & 307946 & 500548 
                            & 362198 & 430744.6\\ \hline
            
                            \cellcolor[HTML]{FFFE65}{\color[HTML]{000000}}
                            B (César) & 988937.98 & 734887.83 & 489798.29 
                            & 936797.44 & 644343.29 & 758952.97\\ \hline
            
                            \cellcolor[HTML]{34FF34}{\color[HTML]{000000}} 
                            C (Edgar) & 1211505.83 & 1027960.93 & 870849.42 
                            & 1361435 & 951458 & 1084641.84\\ \hline
            
                            \cellcolor[HTML]{9698ED}{\color[HTML]{000000} 
                            D(Sandra)} & 2115109.73 & 1710230.42 & 1342941.46 
                            & 2133463 & 1492232 & 1758795.32 \\ \hline
                        \end{tabular}
                    \end{table} 
                }
            \end{itemize}
            
        }
        \item {
            Calcula los tiempos normalizados y obtén la medida de tendencia 
            central adecuada de cada una de las computadoras. \\
            Los datos estarán normalizados respecto a la computadora A.

            \begin{table}[H]
                \caption*{Datos de tiempo de respuesta normalizados respecto a A}
                \begin{tabular}{|l|l|l|l|l|l|l|l|l|l|}
                \toprule
                Pc / Tareas 
                & \cellcolor[HTML]{DAE8FC}build-mplayer 
                & \cellcolor[HTML]{DAE8FC}build-php 
                & \cellcolor[HTML]{DAE8FC}gzip 
                & \cellcolor[HTML]{DAE8FC}dcraw  
                & \cellcolor[HTML]{DAE8FC}flac 
                & \cellcolor[HTML]{DAE8FC}gnupg 
                & \cellcolor[HTML]{DAE8FC}mafft 
                & \cellcolor[HTML]{DAE8FC}mrbayes 
                & \cellcolor[HTML]{DAE8FC}media geo \\ \hline
    
                \cellcolor[HTML]{F8A102}{\color[HTML]{000000}} 
                A (Alan) & 1 & 1 & 1 & 1 & 1 & 1 & 1 & 1 & 1 \\ \hline
    
                \cellcolor[HTML]{FFFE65}{\color[HTML]{000000}}
                B (César) & 0.01382 & 0.65872 & 0.80841 & 0.79981 & 1.1323 
                & 1.1185 & 0.89001 & 0.7402 & 0.51456 \\ \hline
    
                
    
                \cellcolor[HTML]{34FF34}{\color[HTML]{000000}}
                C (Edgar) & 0.0082638 & 0.35472 & 0.61475 & 0.33751 & 0.45100 
                & 0.50925 & 0.40557 & 0.29936 & 0.25332\\ \hline
    
                
    
                \cellcolor[HTML]{9698ED}{\color[HTML]{000000} 
                D(Sandra)} & 0.0067247 & 0.26197 & 0.49116 & 0.26638 & 0.30527 
                & 0.38240 & 0.31584 & 0.24528 & 0.19493 \\ \hline
    
                
                \end{tabular}
            \end{table}
    
            \begin{table}[H]
                \caption*{Datos de rendimiento normalizados respecto a A}
                \begin{tabular}{|l|l|l|l|l|l|l|}
                \toprule
                    Pc / Tareas 
                    & \cellcolor[HTML]{DAE8FC}redis(LPOP) 
                    & \cellcolor[HTML]{DAE8FC}redis(SADD) 
                    & \cellcolor[HTML]{DAE8FC}redis(LPUSH) 
                    & \cellcolor[HTML]{DAE8FC}redis(GET)
                    & \cellcolor[HTML]{DAE8FC}redis(SET) 
                    & \cellcolor[HTML]{DAE8FC}media geo \\ \hline
    
                    \cellcolor[HTML]{F8A102}{\color[HTML]{000000}} 
                    A (Alan) & 1 & 1 & 1 & 1 & 1 & 1 \\ \hline
    
                    \cellcolor[HTML]{FFFE65}{\color[HTML]{000000}}
                    B (César) & 1.787170564 & 1.710326199 & 1.590533048 
                    & 1.871543668 & 1.778980806 & 1.745146036\\ \hline
    
    
                    \cellcolor[HTML]{34FF34}{\color[HTML]{000000}} 
                    C (Edgar) & 2.189386595 & 2.392403899 & 2.827928988 
                    & 2.719889002 & 2.626900204 & 2.54052833\\ \hline
    
                    \cellcolor[HTML]{9698ED}{\color[HTML]{000000} 
                    D(Sandra)} & 3.822344702 & 3.980269877 & 4.36096413 
                    & 4.262254569 & 4.119934401 & 4.104600691 \\ \hline
    
                 
                \end{tabular}
            \end{table} 
        }
        \item {
            Plantea un caso de uso para una computadora. De acuerdo a los 
            requerimientos del usuario, pondera loas resultados de las pruebas 
            y obtén la medida de desempeño de cada una de las computadoras de tu
            equipo. \\
            Como caso de uso se propone un desarrollador de software que necesita
            una computadora personal. \\
            En este caso no es tan relevante que tenga buen desempeño en 
            analisis de datos, por lo que el analisis bayesiano y el analisis 
            genético deberían tener poco peso. \\
            Herramientas para manipulación de media y archivos son medianamente 
            importante, pues aunque son indispensables es suficiente con que 
            funciones aceptablemente. \\
            La parte más importante serían las pruebas de compilación, pues es 
            básicamente para lo que se va a usar la computadora. \\
            En la parte de desempeño, no se requiere que se pueda manejar grandes 
            cantidades de peticiones, pues el equipo no está pensado para usarse
            como servidor. Aún así, sería conveniente que tenga un buen desempeño
            al realizar peticiones, pues esto puede ser útil al realizar pruebas
            de conexión entre la interfaz de usuario y algún servidor.

            \begin{table}[H]
                \caption*{Pesos para los tiempos de respuesta}
                \begin{center}
                    \begin{tabular}{|l|l|l|l|l|l|l|l|l|l|}
                        \toprule
                        Pc / Tareas 
                        & \cellcolor[HTML]{DAE8FC}build-mplayer 
                        & \cellcolor[HTML]{DAE8FC}build-php 
                        & \cellcolor[HTML]{DAE8FC}gzip 
                        & \cellcolor[HTML]{DAE8FC}dcraw  
                        & \cellcolor[HTML]{DAE8FC}flac 
                        & \cellcolor[HTML]{DAE8FC}gnupg 
                        & \cellcolor[HTML]{DAE8FC}mafft 
                        & \cellcolor[HTML]{DAE8FC}mrbayes \\ \hline
            
                        \cellcolor[HTML]{F8A102}{\color[HTML]{000000}} 
                        Pesos & 0.25 & 0.25 & 0.1 & 0.1 & 0.1 & 0.1 & 0.05 
                        & 0.05 \\ \hline
        
                        \end{tabular}
                \end{center}
            \end{table}

            \begin{table}[H]
                \caption*{Medidas de tendencia de tiempo de respuesta ponderadas}
                \begin{center}
                    \begin{tabular}{|l|l|l|l|l|l|l|l|l|l|}
                        \toprule
                        Pc / Tareas 
                        & \cellcolor[HTML]{DAE8FC}media arm \\ \hline
            
                        \cellcolor[HTML]{F8A102}{\color[HTML]{000000}} 
                        A (Alan) & 108.736416907889 \\ \hline
            
                        \cellcolor[HTML]{FFFE65}{\color[HTML]{000000}}
                        B (César) & 19.3724720236483 \\ \hline
            
                        \cellcolor[HTML]{34FF34}{\color[HTML]{000000}}
                        C (Edgar) & 11.0564922536646 \\ \hline
            
                        \cellcolor[HTML]{9698ED}{\color[HTML]{000000} 
                        D(Sandra)} & 8.81993499466002 \\ \hline
            
                        \end{tabular}
                \end{center}
            \end{table}

            \begin{table}[H]
                \caption*{Pesos para las pruebas de desempeño}
                \begin{center}
                    \begin{tabular}{|l|l|l|l|l|l|l|}
                        \toprule
                            Pc / Tareas 
                            & \cellcolor[HTML]{DAE8FC}redis(LPOP) 
                            & \cellcolor[HTML]{DAE8FC}redis(SADD) 
                            & \cellcolor[HTML]{DAE8FC}redis(LPUSH) 
                            & \cellcolor[HTML]{DAE8FC}redis(GET)
                            & \cellcolor[HTML]{DAE8FC}redis(SET) \\ \hline
            
                            \cellcolor[HTML]{F8A102}{\color[HTML]{000000}} 
                            Pesos & 0.25 & 0.15 & 0.15 & 0.15 & 0.3 \\ \hline
        
                        \end{tabular}
                \end{center}
            \end{table} 

            \begin{table}[H]
                \caption*{Medidias de tendencias de desempeño ponderadas}
                \begin{center}
                    \begin{tabular}{|l|l|l|l|l|l|l|}
                        \toprule
                            Pc / Tareas 
                            & \cellcolor[HTML]{DAE8FC}media arit \\ \hline
            
                            \cellcolor[HTML]{F8A102}{\color[HTML]{000000}} 
                            A (Alan) & 432723.55 \\ \hline
            
                            \cellcolor[HTML]{FFFE65}{\color[HTML]{000000}}
                            B (César) & 764760.016 \\ \hline
            
            
                            \cellcolor[HTML]{34FF34}{\color[HTML]{000000}} 
                            C (Edgar) & 1077350.66\\ \hline
            
                            \cellcolor[HTML]{9698ED}{\color[HTML]{000000} 
                            D(Sandra)} & 1754442.2645 \\ \hline
                    
                        \end{tabular}
                \end{center}
            \end{table} 
        }
    \end{enumerate}

    \section{Preguntas}

    \begin{enumerate}
        \item {
            ¿Cuál computadora tiene el mejor tiempo de ejecución? ¿Porqué 
        factor es mejor computadora comparado a la peor? \\
        El tiempo de ejecución de la computadora A es 0.19493 peor que la 
        computadora D.
        }
        \item {
            ¿Cuál computadora tiene el mejor desempeño? ¿Porqué 
        factor es mejor la computadora comparado a la peor? \\
        El rendimiento de la computadora A es 4.104600691 mejor veces que la 
        computadora B.
        }
        \item {
            De acuerdo a la referencia ¿cuál computadora tiene el mejor desempeño
            y cuál tiene el peor desempeño? \\
            La computadora D tiene el mejor desempeño y la A tiene el mejor 
            desempeño.
        }
        \item {
            ¿Cuál computadora tiene el mejor desempeño para el usuario planteado
            en el caso de uso? \\
            La computadora D tiene el mejor desempeño y la A tiene el mejor 
            desempeño.
        }
        \item {
            De los atributos de cada máquina, ¿cuáles resultan determinantes en 
            la pérdida o ganancia de desempeño? \\
            Los atributos que parecen más influyentes son la cantidad de hilos 
            de ejecución y la cantidad de memoria RAM, y en menor medida la 
            velocidad del procesador y de la tarjeta gráfica.
        }
    \end{enumerate}

\end{document}